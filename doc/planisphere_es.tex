% planisphere_es.tex
%
% The LaTeX code in this file brings together into a single document the
% various components of the model planisphere.
%
% Copyright (C) 2014-2024 Dominic Ford <https://dcford.org.uk/>
%
% This code is free software; you can redistribute it and/or modify it under
% the terms of the GNU General Public License as published by the Free Software
% Foundation; either version 2 of the License, or (at your option) any later
% version.
%
% You should have received a copy of the GNU General Public License along with
% this file; if not, write to the Free Software Foundation, Inc., 51 Franklin
% Street, Fifth Floor, Boston, MA  02110-1301, USA

% ----------------------------------------------------------------------------

\documentclass[a4paper,onecolumn,10pt]{article}
\usepackage[dvips]{graphicx}
\usepackage{fancyhdr,url}
\usepackage[utf8]{inputenc}
\usepackage{parskip}
\usepackage[pdftitle={Crea tu propio planisferio}, pdfauthor={Dominic Ford}, pdfsubject={Crea tu propio planisferio}, pdfkeywords={Crea tu propio planisferio}, colorlinks=true, linkcolor=blue, citecolor=blue, filecolor=blue, urlcolor=blue]{hyperref}
\usepackage[T1]{fontenc}
\renewcommand{\familydefault}{\sfdefault}
\pagestyle{fancy}

\lhead{\it Crea tu propio planisferio}
\chead{}
\rhead{\thepage}
\lfoot{}\rfoot{}
\cfoot{\bf\footnotesize\copyright\ 2014--2024 Dominic Ford. Distribuido bajo GNU General Public License, versión 3. Documento descargado de \url{https://in-the-sky.org/planisphere/}}

\fancypagestyle{plain}{%
\fancyhf{} % clear all header and footer fields
\renewcommand{\headrulewidth}{0pt}
\renewcommand{\footrulewidth}{0pt}}

\title{Crea tu propio planisferio}
\author{Dominic Ford}
\date{2014--2024}

\begin{document}
\maketitle
\setcounter{footnote}{1}

Un planisferio es un dispositivo portátil simple que muestra un mapa de las estrellas visibles
en el cielo nocturno en cualquier momento en particular. Al rotar la rueda de estrellas se muestra
cómo se mueven las estrellas en el cielo durante la noche, y qué constelaciones son visibles
en diferentes momentos del año.

Aquí presento un kit que puedes descargar e imprimir para crear tu propio
planisferio de papel o cartón.

El diseño del planisferio depende de la ubicación geográfica en la que será usado,
ya que distintas estrellas son visibles desde diferentes lugares. He creado kits
para ser usados en una gran variedad de latitudes, los cuales pueden ser descargados en:

\url{https://in-the-sky.org/planisphere/}

El planisferio presentado en este documento fue diseñado para ser usado en la latitud de
\input{tmp/lat}.

\section*{Lo que necesitas}

\begin{itemize}
\item Dos hojas de papel A4 o, preferentemente, cartulina o cartón fino.
\item Tijera.
\item Broche de dos puntas (mariposa).
\item Opcional: una hoja de plástico transparente, por ejemplo el acetato usado con proyectores
\item Opcional: Un poco de cola o pegamento.
\end{itemize}

\section*{Instrucciones de armado}

{\bf Paso 1} -- Los planisferios son un poco diferentes dependiendo de dónde estás.
El planisferio preparado en este documento está diseñado para ser usado en cualquier
lugar de la Tierra que esté dentro de algunos grados de la latitud \input{tmp/lat}.
Si vives en otro lugar debes descargar un kit alternativo en

\url{https://in-the-sky.org/planisphere/}

{\bf Paso 2} -- Imprime las páginas al final de este archivo PDF que incluyen la
rueda de estrellas y el cuerpo del planisferio, en dos hojas de papel separadas, o preferentemente
en cartulina o cartón fino.

{\bf Paso 3} -- Corta cuidadosamente la rueda de estrellas y el cuerpo del planisferio. También
corta el área sombreada del cuerpo del planisferio y, opcionalmente, el visor transparente. Si estás
usando cartón, puedes marcar cuidadosamente el cuerpo del planisferio sobre la línea punteada, para
facilitar doblarla más adelante.

{\bf Paso 4} -- La rueda de estrellas tiene un pequeño círculo en el centro y el
cuerpo del planisferio tiene un pequeño círculo correspondiente en la parte inferior.
Haz un pequeño agujero de aproximadamente 2 mm de diámetro en cada uno. Puedes usar
la punta de un compás y agrandar el agujero girando el papel.

{\bf Paso 5} -- Coloca un broche de dos puntas en el medio de la rueda de estrellas,
con la cabeza del broche ubicada en la cara impresa de la rueda. Luego pasa el broche
por el agujero en el cuerpo del planisferio, con el lado impreso hacia las puntas del
broche. Finalmente doble las puntas del broche para ajustar las dos hojas juntas.

{\bf Paso 6 (Opcional)} -- Si imprimiste la página final del PDF en una hoja transparente,
ahora debes pegar esta guía de líneas sobre el visor cortado en el cuerpo del planisferio.

{\bf Paso 7} --Dobla el cuerpo del planisferio a lo largo de la línea punteada, de
manera que el frente de la rueda de estrellas aparezca a través del visor cortado
en el cuerpo del planisferio.

{\bf Felicitaciones, tu planisferio está listo para ser usado!}

\section*{Cómo usar tu planisferio}

Gira la rueda de estrellas hasta encontrar el punto del borde donde está la fecha
del día, y alínealo con la hora actual. El visor ahora muestra todas las constelaciones
visibles en el cielo.

Ve afuera y apunta al Norte o Sur dependiendo de tu hemisfério. Sosteniendo el planisferio
hacia el cielo, las estrellas indicadas en el borde inferior del visor deberían coincidir
con las que ves en el cielo frente a ti

Gira hacia el Este o el Oeste, y gira el planisferio hasta que la palabra "Este" u
"Oeste" esté en la parte inferior del visor. Nuevamente, las estrellas en el borde
inferior deberían coincidir con lo que ves en el cielo frente a ti

Si imprimiste la red de líneas guía de altitud y azimut en plástico transparente,
esta te permitirá ver qué tan alto se encuentran los objetos en el cielo, y en qué
dirección. Los círculos están dibujados en altitudes de 10, 20, 30, ..., 80 grados
sobre el horizonte. Como referencia, una distancia de 10 grados equivale a la
medida de una palma con el brazo extendido. Las líneas curvas son líneas verticales
uniendo puntos del horizonte con el punto exactamente sobre tu cabeza. Están
dibujadas en los puntos cardinales: S, SSE, SE, ESE, E, etc.

\section*{Planisferios personalizados}

Este kit fue diseñado usando una colección de scripts de Python y
la librería de gráficos pycairo. Si deseas personalizar tu planisferio
estás invitado a descargar los scripts desde mi cuenta de GitHub y
modificarlos, siempre y cuando menciones la fuente:

\url{https://github.com/dcf21/planisphere}

\section*{Licencia}

Como todo el contenido en {\tt In-The-Sky.org}, estos kits de planisferios son
\copyright\ Dominic Ford. Sin embargo, todo en {\tt In-The-Sky.org} es provisto
para el beneficio de astrónomos amateurs en todo el mundo, y eres invitado a
modificar y/o redistribuir cualquier material de este sitio, con las siguientes
condiciones: (1) Cualquier item que tenga un texto de copyright asociado {\bf debe}
incluir el texto {\bf sin modificaciones} en su versión redistribuida, (2) Tu
{\bf debes} citarme a mí, Dominic Ford, como autor original y dueño de los derechos
autorales, (3) Tu {\bf no podrás} lucrar con la reproducción del material de este
sitio, {\bf a menos que} seas una institución de caridad registrada cuyo objetivo
expreso sea el avance de la ciencia astronómica, {\bf o} tengas el permiso escrito
del autor.

\newpage

\centerline{\includegraphics{tmp/starwheel}}

\vspace{1cm}
La rueda de estrellas central del planisferio, que debe ser encajada dentro del soporte doblado.

\newpage
\thispagestyle{empty}
\vspace*{-3.0cm}
\centerline{\includegraphics{tmp/holder}}
\newpage

\centerline{\includegraphics{tmp/altaz}}

\vspace{1cm}
Esta guía puede ser impresa opcionalmente en una hoja transparente, y pegada en la
ventana recortada en el cuerpo del planisferio para mostrar la altitud de los objetos
en el cielo y sus direcciónes.

\end{document}

