% planisphere.tex
%
% The LaTeX code in this file brings together into a single document the
% various components of the model planisphere.
%
% Copyright (C) 2014-2024 Dominic Ford <https://dcford.org.uk/>
%
% This code is free software; you can redistribute it and/or modify it under
% the terms of the GNU General Public License as published by the Free Software
% Foundation; either version 2 of the License, or (at your option) any later
% version.
%
% You should have received a copy of the GNU General Public License along with
% this file; if not, write to the Free Software Foundation, Inc., 51 Franklin
% Street, Fifth Floor, Boston, MA  02110-1301, USA

% ----------------------------------------------------------------------------

\documentclass[a4paper,onecolumn,10pt]{article}
\usepackage[dvips]{graphicx}
\usepackage{fancyhdr,url}
\usepackage[utf8]{inputenc}
\usepackage{parskip}
\usepackage[pdftitle={Pasigaminkite planisferą}, pdfauthor={Dominic Ford}, pdfsubject={Pasigaminkite planisferą}, pdfkeywords={Pasigaminkite planisferą}, colorlinks=true, linkcolor=blue, citecolor=blue, filecolor=blue, urlcolor=blue]{hyperref}
\usepackage[T1]{fontenc}
\renewcommand{\familydefault}{\sfdefault}
\pagestyle{fancy}

\lhead{\it Pasigaminkite planisferą}
\chead{}
\rhead{\thepage}
\lfoot{}\rfoot{}
\cfoot{\bf\footnotesize\copyright\ 2014--2024 Dominic Ford. Platinama pagal GNU Bendrosios viešosios licencijos 3 versiją. Daugiau info \url{https://in-the-sky.org/planisphere/}}

\fancypagestyle{plain}{%
\fancyhf{} % clear all header and footer fields
\renewcommand{\headrulewidth}{0pt}
\renewcommand{\footrulewidth}{0pt}}

\title{Pasigaminkite sukamąjį žvaigždėlapį (planisferą)}
\author{Dominic Ford}
\date{2014--2024}

\begin{document}
\maketitle
\setcounter{footnote}{1}

Planisfera - tai sukamasis žvaigždėlapis, kuriame yra pavaizduotos žvaigždės
naktiniame danguje tam tikru metu. Sukant žvaigždžių diską, jis rodo
kaip žvaigždės juda dangumi, kokios žvaigždės ir
kokie žvaigždynai yra matomi atitinkamoje platumoje skirtingais metų laikais.Platumos paklaida iki 5 laipsnių nėra reikšminga.

Čia pateikiamas planisferos brėžinių rinkinys, kurį galite atsisiųsti ir atsispausdinti, kad pasigamintumėte savo planisferą iš popieriaus ar kartono.

\section*{Reikalingos priemonės}

\begin{itemize}
\item Du A4 formato popieriaus lapai, arba storesnis kartonas.
\item Žirklės.
\item Kniedė, M2 arba M3 varžtelis (su veržle ir poveržle) arba viela centrinei ašiai.
\item Nebūtina: skaidraus plastiko skaidrė, skirta jūsų spausdintuvui.
\item Nebūtina: Klijai.
\end{itemize}

\section*{Surinkimo instrukcija}

{\bf 1-as žingsnis.} -- Planisferos skiriasi, priklausomai nuo to, kokioje platumoje
gyvenate. Ši planisfera yra skirta naudoti bet kurioje vietovėje, iki 5 platumos laipsnių nutolusioje nuo \input{tmp/lat}. Pasirinkite sugeneruotų planisferos brėžinių rinkinį, tinkamą savo platumai arba atsisųskite iš

\centerline{\tt https://in-the-sky.org/planisphere}

{\bf 2-as žingsnis.} -- Atsispausdinkite šio PDF failo gale esančius puslapius, kuriuose
yra žvaigždžių diskas ir planisferos pagrindas. Geriau tinka storesnis popierius
arba galite paklijuoti atspausdintas planisferos dalis ant plono kartono.

{\bf 3-as žingsnis.} -- Atsargiai išpjaukite žvaigždžių diską ir
planisferos pagrindą. Taip pat iškirpkite pilką planisferos pagrindo plotą ir
jei turite, iškirpkite linijų tinklelį, kurį atspausdinote ant permatomo
plastiko. Jei naudojate kartoną, galite prabraukti neaštriu
daiktu planisferos pagrindą išilgai skersinės linijos, kad būtų lengviau perlenkti.

{\bf 4-as žingsnis.} -- Žvaigždžių disko centre yra mažas apskritimas,
planisferos pagrindo apatinėje dalyje yra toks pat mažas apskritimas. Padarykite po skylutę
(pagal turimos ašies skersmenį) žvaigždžių diske ir planisferos pagrinde. Naudokite popieriaus perforatorių, grąžtelį arba kitą smailą įrankį. Jeigu reikia, padidinkite skylutę, kad atitiktų ašies skersmenį.

{\bf 5-as žingsnis.} -- Abi detales sujunkite įkišę sukimosi ašį (kniedę, varžtelį ar vielą). Užlenkite vielą iš abiejų pusių (ant varžtelio uždėkite poveržlę ir užsukite veržlę, kniede - užtvirtinkite) taip, kad žvaigždžių diskas ir pagrindas būtų prisiglaudę vienas ant kito.


{\bf 6-as žingsnis (nebūtinas)} -- Jei atspausdinote paskutinį PDF failo puslapį ant permatomo plastiko lapo, priklijuokite jį ant žiūrėjimo langelio, kurį išpjovėte iš planisferos pagrindo.


{\bf 7-as žingsnis} -- Planisferos pagrindą sulenkite išilgai skersinės linijos,
taip, kad žvaigždžių diskas būtų viduje ir matytųsi pro planisferos pagrindo langelį, kurį išpjovėte.


{\bf Sveikinu, jūsų planisfera parengta naudoti!}

\section*{Kaip naudoti planisferą}

Sukite žvaigždžių diską tol, kol rasite tašką, kurio krašte yra atitinkama data.
Sulygiuokite šį tašką su atitinkamu laiku. Peržiūros lange bus matomi tuo metu danguje 
matomi žvaigždynai.

Išeikite į lauką ir atsigręžkite į šiaurę. Laikydami planisferą akių lygyje, užrašu „Šiaurė“ į apačią, pažvelkite į žvaigždes. Planisferos lango apatinėje dalyje matomos žvaigždės turėtų sutapti su tomis, kurias kurias matote danguje priešais save.



Pasisukite į rytus arba vakarus ir pasukite planisferą taip, kad lango apačioje būtų užrašas \guillemotleft Rytai\guillemotright\ arba \guillemotleft Vakarai\guillemotright\ . Apatinėje planisferos lango dalyje esančios žvaigždės turėtų atitikti tas, kurias matote priešais save danguje.

Jei atsispausdinote permatomą aukščio ir azimuto linijų tinklelį, galite jį naudoti nustatydami, kaip aukštai ir kokia kryptimi danguje matomi objektai. Apskritimai brėžiami 10, 20, 30... 80 laipsnių aukštyje virš horizonto. Palyginimui 10$^\circ$ atstumas maždaug atitinka jūsų kumščio plotį, kai ištiesiate ranką. Lenktos linijos - tai vertikalės, jungiančios taškus horizonte su zenito tašku, tiesiai virš jūsų galvos. Jos išdėstytos ties pagrindiniais taškais P, PPR, PR, RPR, R ir t. t.


\section*{Autorinės teisės}

Kaip ir visos svetainės In-The-Sky.org, šios planisferos autorinės teisės priklauso Dominicui Fordui. Tačiau In-The-Sky. org yra prieinamas visas astronomijos mėgėjams visame pasaulyje, ir jūs galite laisvai keisti ir (arba) platinti bet kurią šios svetainės dalį, laikydamiesi toliau nurodytų sąlygų: (1) bet kokią medžiagą, prie kurios pridėtas pranešimas apie autorių teises, privalote įtraukti į platinamą versiją be pakeitimų, (2) privalote nurodyti Dominiką Fordą, kaip autorių ir autorių teisių turėtoją (3) negalite gauti pelno iš šios svetainės turinio atgaminimo, išskyrus atvejus, kai esate oficiali pelno nesiekianti organizacija, kurios konkretus tikslas - mokslinė pažanga astronomijos srityje, arba turite raštišką autoriaus leidimą.

\newpage

\centerline{\includegraphics{tmp/starwheel}}

\vspace{1cm}
Planisferos žvaigždžių diskas, kuris turėtų būti įdėtas į sulenktą planisferos pagrindą.

\newpage
\thispagestyle{empty}
\vspace*{-3.0cm}
\centerline{\includegraphics{tmp/holder}}
\newpage

\centerline{\includegraphics{tmp/altaz}}

\vspace{1cm}
Šį linijų tinklelį galima atspausdinti ant skaidraus plastiko ir priklijuoti ant planisferos pagrinde iškirpto langelio, kad būtų galima parodyti dangaus objektų aukščius ir jų azimutą.

\end{document}

